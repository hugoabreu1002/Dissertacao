\chapter*{Agradecimentos}

Agradeço primeiramente aos professores do Programa de Pós Graduação em Engenharia de Sistemas da POLI, por não se esforçados para manter o compromisso com os discentes e oferta de disciplinas em um ano tão difícil como o de 2020. O contexto da Pandemia realmente dificultou muitas coisas, porém com a ajuda do meu orientador Manoel Marinho e do meu coorientador Fausto Lorenzato o trabalho pode ter continuidade com sucesso. No mais agradeço a meus pais pelo suporte.

Este trabalho é fruto do projeto de pesquisa e desenvolvimento intitulado "Arranjo técnico para aumento da confiabilidade e segurança elétrica aplicando armazenamento de energia por baterias e sistemas fotovoltaicos ao serviço auxiliar de subestações 230/500 kV", proveniente da Chamada Pública - P&D+I Nº 02/2019 com financiamento da Companhia Hidro Elétrica do São Francisco (CHESF). Trata-se de uma iniciativa no âmbito do Programa de Pesquisa e Desenvolvimento Tecnológico do Setor de Energia Elétrica da ANEEL, sob execução da Universidade de Pernambuco (UPE), Instituto de Tecnologia Edson Mororó Moura – ITEMM e Fundação Parque Tecnológico Itaipu – PTI.
