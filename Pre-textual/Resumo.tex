\chapter*{Resumo}

A estimativa de variáveis físicas pode ser feita através de modelagens numéricas variadas. Uma das formas úteis se dá pela analise e modelagem de séries temporais. Dentro do contexto da geração de energias limpas, são feitas previsões de radiação solar, para usinas fotovoltaicas e velocidade do vento para usinas eólicas. Modelos lineares de séries temporais como ARIMA são muito utilizados, porém muito se avançou utilizando técnicas de \textit{machine learning} para melhorar os resultados, que podem ser utilizadas em conjunto com modelos lineares, resultando em modelos híbridos. Neste trabalho é apresentada uma nova forma de automatizar a modelagem SARIMAX a partir do uso em conjunto dos algoritmos de otimização PSO e ACO, levando em consideração a sazonalidade e possíveis variáveis exógenas disponíveis. Também são apresentados 2 modelos híbridas distintos que possuem MLPs como elementos principais. Um protocolo foi utilizado para obtenção dos resultados, que foram obtidos para tais modelos que se mostraram muito promissores para utilização em sistemas automáticos de previsão de radiação.


\noindent\par\vspace{1em}
\noindent\textbf{Palavras-chave: Radiação solar, séries temporais, aprendizado de máquina, otimização} 