\chapter{Modelagem de Séries Temporais}
\label{cap:series_temp}

\section{Séries temporais}


Series temporais se tratam de um conjunto de observações orientadas no tempo \cite{brockwell2002introduction}. Diferentemente de sinais, series temporais são sempre discretas.

O fato de series temporais serem denominadas dessa forma está na correlação temporal de cada um dos seus pontos. A correlação temporal significa que cada observação pode ser linearmente modelada por observações passadas. 

\section{Modelos Lineares}

\subsection{ARIMA}

Dada uma série temporal definida como $X_t$ sendo $t$ o índice inteiro de cada observação, o modelo ARMA$(p^{'},q)$ é dado pela Equação \ref{eq:arma_model} abaixo, o parâmetro $\alpha_i$ é responsável pela parte auto regressiva do modelo, enquanto que $\theta_i$ pela média móvel. O termo $\varepsilon_t$ representa o erro que é assumido como uma variável aleatória IID, amostrada a partir de uma distribuição normal com média zero \cite{brockwell2002introduction}. 

\begin{equation}
\label{eq:arma_model}
    X_t-\alpha_1X_{t-1}- \dots -\alpha_{p'}X_{t-p'} = \varepsilon_t + \theta_1 \varepsilon_{t-1} + \cdots +\theta_q \varepsilon_{t-q}
\end{equation}

Uma outra forma de descrever o modelo ARMA é obtida ao se utilizar o operador $L$ que representa os \textit{Lags}, como sendo índices do passado em relação a uma variável, como na Equação \ref{eq:arma_model_L}.

\begin{equation}
\label{eq:arma_model_L}
    \left(1 - \sum_{i=1}^{p'} \alpha_i L^i\right) X_t=\left(1 + \sum_{i=1}^q \theta_i L^i\right) \varepsilon_t
\end{equation}

Assumindo que o polinômio $\left( 1 - \sum_{i=1}^{p'} \alpha_i L^i \right)$ possui uma raiz unitária como um fator de  $(1-L)$ de multiplicidade $d$, então a Equação \ref{eq:arma_model_L} pode ser reescrita como \ref{eq:arma_model_alpta2phi}.

\begin{equation}
\label{eq:arma_model_alpta2phi}
    \left(1 - \sum_{i=1}^{p'} \alpha_i L^i\right)=\left(1 - \sum_{i=1}^{p'-d} \varphi_i L^i\right)\left(1 - L \right)^d
\end{equation}

O modelo ARIMA$(p,d,q)$ é uma derivação do modelo ARMA ao expressar a fatoração polinomial mostrada anteriormente, substituindo \ref{eq:arma_model_alpta2phi} em \ref{eq:arma_model_L} e definindo $p=p^{'} - d$, o que leva a equação \ref{eq:arima_model}.

\begin{equation}
\label{eq:arima_model}
    \left( 1 - \sum_{i=1}^p \varphi_i L^i \right) (1-L)^d X_t = \left( 1 + \sum_{i=1}^q \theta_i L^i \right) \varepsilon_t \
\end{equation}

A equação \ref{eq:arima_model} pode ser ainda simplificada da forma \ref{eq:arima_model_simple}, esta simplificação ajuda a na modelagem SARIMAX que seguirá na próxima seção \ref{subsec:sarimax}.

\begin{equation}
\label{eq:arima_model_simple}
    \phi_p(L)(1-L)^d X_t = c + \theta_q(L)\varepsilon_t
\end{equation}

Em que, $\phi_p(L)$ se torna o termo auto regressivo de ordem $p$, $c$ é uma constante e $\theta_q(L)\varepsilon_t$ o termo de médias móveis de ordem $q$, $d$ é a ordem de diferenciação. 

A teoria dos modelos ARIMA teve sua ampla aplicação por conta dos esforços dos trabalhos de Box et al. \cite{box2011time}, que desenvolveu um método sistemático, iterativo e prático de construção de modelos \cite{ramos2015performance}, dividido em três etapas, identificação do modelo, estimativa de parâmetro e diagnóstico do modelo.

\subsection{SARIMAX}
\label{subsec:sarimax}

A partir da equação \ref{eq:arima_model_simple} que descreve um modelo ARIMA não sazonal é possível chegar a modelagem SARIMAX, que leva em conta a sazonalidade e também variáveis exógenas. Primeiramente, o modelo SARIMA, que leva em consideração apenas a sazonalidade é representado pela equação \cite{arunraj2016application, de2018modelo}

\begin{equation}
\label{eq:sarima_model}
    \phi_p(L)\Phi_P(L^S)(1-L)^d(1-L^S)^D X_t = c + \theta_q(L)\Theta_Q(L^S)\varepsilon_t
\end{equation}

Em que, $\Phi_P(L^S)$ se torna o termo auto regressivo sazonal de ordem $P$, $S$ é a frequência de sazonalidade, deve ser 12 para dados mensais e 24 para dados horários. $D$ é a quantidade de diferenciações para a parte sazonal do modelo e $\Theta_Q(L^S)$ se trata do termo de médias móveis sazonal.

O modelo SARIMA tem capacidade de lidar séries temporais estacionárias e não estacionárias com elementos sazonais \cite{arunraj2016application}. Com base no comportamento das séries temporais, dados exógenos podem ter potencial impacto em melhorar as estimativas do modelo. Portanto, é importante considerar variáveis externas, que fornecem respostas significativas para os dados remotos. O modelo que lida com variáveis externas é chamado SARIMAX. Neste modelo, variáveis exógenas $X_t$ podem ser modeladas com uma regressão linear múltipla \cite{arunraj2016application,  elamin2018modeling}, com pesos $\beta$. O termo $w_t$ trata da série residual restante e pode ser representado a partir do modelo SARIMA, como mostra a equação \ref{eq:sarimax_model}.

\begin{equation}
\label{eq:sarimax_model}
    \begin{gathered}
        Z_t = c + \beta X_t +w_t\\
        w_t = \frac{\theta_q(L)\Theta_Q(L^S)}{ \phi_p(L)\Phi_P(L^S)(1-L)^d(1-L^S)^D}\varepsilon_t
    \end{gathered}
\end{equation}

\section{Métricas de Performance}

As métricas de performance são necessárias para identificar se os modelos se comportam de melhor forma. As métricas que serão apresentadas a seguir são comumente utilizadas seja para zerar e otimizar um modelo ARIMA/SARIMAX, como no caso do AIC, como para comparar modelagens híbridas, no caso de MAPE, MAE e RMSE.

O critério de Akaike (AIC) estima a perda de informação quando uma distribuição de probabilidade $f$ associada aos dados reais é aproximada por uma distribuição de probabilidade $g$, obtida a partir de algum modelo. A medida de discrepância entre $f$ e $g$ $I(f,g)$ é igual a medida de informação de Kullback–Leibler \cite{wagenmakers2004aic}.

Por fim é definido como na equação \ref{eq:aic}, em que $L$ é o modelo candidato com máxima verossimilhança, determinado a partir de ajustes de $k$, que são parâmetros do modelo, de uma forma que seja maximizada a probabilidade de que $f$ seja equivalente $g$ \cite{wagenmakers2004aic}.

\begin{equation}
\label{eq:aic}
    AIC = 2k - 2\ln(\hat L)
\end{equation}

As métrica MAPE, MAE e RMSE, dadas pelas equações \ref{eq:MAPE} \ref{eq:MAE} \ref{eq:RMSE}, respectivamente, são utilizadas para avaliar as previsões dos modelos. 

\begin{equation}
\label{eq:MAPE}
    MAPE = \frac{1}{n}\sum_{t=1}^{n} \frac{|Y_t - \hat{Y_t}|}{Y_t}
\end{equation}

\begin{equation}
\label{eq:MAE}
    MAE = \frac{1}{n}\sum_{t=1}^{n} |Y_t - \hat{Y_t}|
\end{equation}

\begin{equation}
\label{eq:RMSE}
    RMSE = \sqrt{\frac{1}{n}\sum_{t=1}^{n} |Y_t - \hat{Y_t}|^2}
\end{equation}

\section{Dados Utilizados} 
\label{cap:dados_prep}

\subsection{Origem}

O Instituto Nacional de Meteorologia disponibiliza dados a partir de diversas estações espalhadas pelo país através de seu Banco de Dados Meteorológicos (BDMEP) \cite{INMET}.

Os dados são adquiridos a partir de estações de medição, havendo as automáticas e as convencionais, estas últimas estão sendo descontinuadas \cite{lima2020comparaccao}. Utilizando os dados do INMET e outras fontes de dados abertos é possível obter criar modelos de interpolação espacial e predição de temporal \cite{mendes2020integraccao}.

Os dados foram obtidos pelo período entre 2008 e 2020, entretanto são trabalhados apenas os dados horários dos últimos 365 dias. Entre os dados presentes, se encontram séries temporais horárias das seguintes variáveis:

\begin{enumerate}
    \item Radiação Global (W/m\textsuperscript{2})
    \item Pressão Atmosférica, máxima e mínima (mb)
    \item \textbf{Precipitação Total} (mm)
    \item \textbf{Temperatura do Ar em Bulbo Seco} (\textdegree{}C)
    \item Temperatura máxima, mínima e de Ponto de Orvalho (\textdegree{}C)
    \item \textbf{Umidade Relativa}, máxima e mínima (\%)
    \item \textbf{Velocidade do Vento e em rajada} (m/s)
    \item Direção do vento (graus)
\end{enumerate}

Estando destacadas as variáveis disponíveis escolhidas como variáveis exógenas para previsão temporal da radiação.

\subsection{Pré-processamento}

Neste trabalho os dados serão utilizados para previsão de séries temporais utilizando os modelos discutidos no capítulo \ref{cap:auto_ml}, desenvolvidos para melhorar a performance de modelos lineares do capítulo \ref{cap:series_temp}. Para isto, é preciso fazer um tratamento inicial na base de dados oriunda do INMET, este tratamento visa adequar as variáveis aos modelos. 

Normalmente, dados obtidos por fontes externas nem sempre estão 100\% corretos, um dos problemas comuns é a presença de lixo, dados vazios ou formato inadequado. No caso dos dados trabalhados do INMET, foi necessário fazer a adequação de formato e complementação de dados faltantes. Além disto, os dados foram trabalhados em escala entre 0 e 1, para isto é feita a divisão pelo máximo valor encontrado na base, para cada variável. No fim da previsão é possível retornar a escala original.

A complementação de dados faltantes se dá a partir do método chamado \textit{forward fill}, que significa que o último valor válido observado é utilizado para substituir valores vazios de cada variável.