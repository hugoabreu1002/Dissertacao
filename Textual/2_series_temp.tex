\chapter{Modelagem de Séries Temporais}
\label{cap:series_temp}

\section{Séries temporais}


Series temporais se tratam de um conjunto de observações orientadas no tempo \cite{brockwell2002introduction}. Diferentemente de sinais, series temporais são sempre discretas.

O fato de series temporais serem denominadas dessa forma está na correlação temporal de cada um dos seus pontos. A correlação temporal significa que cada observação pode ser linearmente modelada por observações passadas. 

\section{Modelos Lineares}

\subsection{ARIMA}

Dada uma série temporal definida como $X_t$ sendo $t$ o índice inteiro de cada observação, o modelo ARMA$(p^{'},q)$ é dado pela Equação \ref{eq:arma_model} abaixo, o parâmetro $\alpha_i$ é responsável pela parte auto regressiva do modelo, enquanto que $\theta_i$ pela média móvel. O termo $\varepsilon_t$ representa o erro que é assumido como uma variável aleatória IID, amostrada a partir de uma distribuição normal com média zero \cite{brockwell2002introduction}. 

\begin{equation}
\label{eq:arma_model}
    X_t-\alpha_1X_{t-1}- \dots -\alpha_{p'}X_{t-p'} = \varepsilon_t + \theta_1 \varepsilon_{t-1} + \cdots +\theta_q \varepsilon_{t-q}
\end{equation}

Uma outra forma de descrever o modelo ARMA é obtida ao se utilizar o operador $L$ que representa os \textit{Lags}, como sendo índices do passado em relação a uma variável, como na Equação \ref{eq:arma_model_L}.

\begin{equation}
\label{eq:arma_model_L}
    \left(1 - \sum_{i=1}^{p'} \alpha_i L^i\right) X_t=\left(1 + \sum_{i=1}^q \theta_i L^i\right) \varepsilon_t
\end{equation}

Assumindo que o polinômio $\left( 1 - \sum_{i=1}^{p'} \alpha_i L^i \right)$ possui uma raiz unitária como um fator de  $(1-L)$ de multiplicidade $d$, então a Equação \ref{eq:arma_model_L} pode ser reescrita como \ref{eq:arma_model_Ld}.

\begin{equation}
\label{eq:arma_model_Ld}
    \left(1 - \sum_{i=1}^{p'} \alpha_i L^i\right)=\left(1 - \sum_{i=1}^{p'-d} \varphi_i L^i\right)\left(1 - L \right)^d
\end{equation}

O modelo ARIMA$(p,d,q)$ é uma derivação do modelo ARMA ao expressar a fatoração polinomial mostrada anteriormente, definindo $p=p^{'} - d$, o que leva a equação \ref{eq:arima_model}.

\begin{equation}
\label{eq:arima_model}
    \left( 1 - \sum_{i=1}^p \varphi_i L^i \right) (1-L)^d X_t = \left( 1 + \sum_{i=1}^q \theta_i L^i \right) \varepsilon_t \
\end{equation}

A equação \ref{eq:arima_model} pode ser ainda simplificada da forma \ref{eq:arima_model_simple}, esta simplificação ajuda a na modelagem SARIMAX que seguirá na próxima seção \ref{subsec:sarimax}.

\begin{equation}
\label{eq:arima_model_simple}
    \Phi_p(L)(1-L)^d X_t = c + \Theta_q(L)\Epsilon_t \
\end{equation}

A teoria dos modelos ARIMA teve sua ampla aplicação por conta dos esforços dos trabalhos de Box et al. \cite{box2011time}, que desenvolveu um método sistemático, iterativo e prático de construção de modelos \cite{ramos2015performance}, dividido em três etapas, identificação do modelo, estimativa de parâmetro e diagnóstico do modelo.

\subsection{SARIMAX}
\label{subsec:sarimax}



\section{Métricas de Performance}

\subsection{AIC}

\subsection{MAPE}

\subsection{MAE}

\subsection{MSE}

%
%%%%%%%%%%%%%%%%%%%%%%%%%%%%%%%%%%%%%%%%%
%
