\chapter{Estado da arte}
\label{cap:estado_da_arte}

Modelos híbridos, em que se utiliza da modelagem linear dada por um ARIMA ou variante já foi muito utilizada e discutida na literatura \cite{zhang2003time, khashei2010artificial, babu2014moving, de2014hybrid, de2016hybrid, domingos2019intelligent}. O intuito destes modelos, chamados híbridos, é de que incrementar a capacidade de adaptação da modelagem linear ARIMA for meio de outros algoritmos de \textit{machine learning} ou \textit{extreme learning machines} \cite{yu2020hybrid}, a combinação de modelos pode ser eficaz em aprimorar o desempenho de previsão \cite{khashei2012new}.

Nos últimos anos, foi comum a implementação de modelos de série temporal linear combinado com \textit{machine learning} para previsões de séries temporais, utilizando ARIMA com RNA \cite{xiong2017hybrid}, SVM \cite{domingos2019intelligent}, LSTM \cite{choi2018stock}. Outros modelagens híbridas podem ser feitas, como \textit{Exponential Smoothing} em conjunto com LSTM \cite{smyl2020hybrid}.

Se é muito utilizado algoritmos de inteligência de enxame no processo de otimização de modelos híbridos. Sendo vistos diversas aplicações utilizando os modelos que são abordados nesse trabalho: Algoritmo Genético (GA) \cite{huang2012hybrid}, algoritmo de Otimização de Enxame de Partículas (PSO) \cite{bagheri2014financial, pradeepkumar2017forecasting}, e algoritmo Ant Colony Optimization (ACO) \cite{shen2013optimal}. Mais especificamente sobre previsão de geração fotovoltaica e irradiação solar, o padrão de utilização de modelos se mantém, \cite{sobri2018solar, wang2019review}, havendo um destaque maior para o uso de \textit{deep learning}.

O uso de algoritmos meta-heurístico para implementação de sistemas automatizados (autoML) pode ser dividido em duas vertentes, sendo otimização de hiper-parâmetros e otimização arquitetural \cite{he2021automl}. Neste trabalho, o protocolo proposto para o autoML consta com as vertentes. Do ponto de vista de implementação, Python é a linguagem mais utilizada em inteligência artificial e \textit{machine learning}, muito devido a facilidade de distribuição de códigos e pacotes, e adoção da comunidade desenvolvedora \cite{blank2020pymoo}. Dentre as bibliotecas mais utilizadas está o \textit{Scikit-Learn}, por conta da sua facilidade de uso, consistência e grande leque de algoritmos disponíveis \cite{scikit-learn, hackeling2017mastering}. 

Além do \textit{Scikit-Learn}, bibliotecas como Keras, que se trata de um envelopamento de auto nível que facilita a construção de redes neurais profundas \cite{jin2019auto} e outros softwares muito utilizados, como WEKA, também constam com camadas de automatização \cite{feurer2020auto}. Neste trabalho, é desenvolvida uma biblioteca específica para series temporais, de acordo com o que é mostrado no capítulo \ref{cap:materiais_e_metodos} sobre materiais e métodos, e também com o protocolo definido na seção \ref{sec:protocolo_resultados}.