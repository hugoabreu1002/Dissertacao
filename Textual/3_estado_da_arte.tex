\chapter{Estado da arte}
\label{cap:estado_da_arte}

Modelos híbridos, em que se utiliza da modelagem linear dada por um ARIMA ou variante já foi muito utilizada e discutida na literatura \cite{zhang2003time, khashei2010artificial, babu2014moving, de2014hybrid, de2016hybrid, domingos2019intelligent}. O intuito destes modelos, chamados híbridos, é de que incrementar a capacidade de adaptação da modelagem linear ARIMA for meio de outros algoritmos de \textit{machine learning} ou \textit{extreme learning machines} \cite{yu2020hybrid}

Nos últimos anos, foi comum a implementação de modelos de série temporal linear combinado com \textit{machine learning} para previsões de séries temporais, utilizando ARIMA com RNA \cite{xiong2017hybrid}, SVM \cite{domingos2019intelligent}, LSTM \cite{choi2018stock}.

Se é muito utilizado algoritmos de inteligência de enxame para no processo de otimização de modelos híbridos. Sendo vistos diversas aplicações utilizando os modelos que são abordados nesse trabalho: Algoritmo Genético (GA) \cite{huang2012hybrid}, algoritmo de Otimização de Enxame de Partículas (PSO) \cite{bagheri2014financial, pradeepkumar2017forecasting}, e algoritmo Ant Colony Optimization (ACO) \cite{shen2013optimal}. Mais especificamente sobre previsão de geração fotovoltaica e irradiação solar, o padrão de utilização de modelos se mantém, \cite{sobri2018solar, wang2019review}, havendo um destaque maior para o uso de \textit{deep learning}.