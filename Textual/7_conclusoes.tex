\chapter{Conclusões}
\label{cap:conclusoes}

A partir dos resultados preliminares obtidos até o momento é possível tirar conclusões sobre os procedimentos definidos e os algoritmos selecionados e descritos para realização do objetivo central de automatizar a criação de modelos que possam ser aplicados dinamicamente. Neste trabalho foram mostrados 3 formas inovadoras que automatização e otimizam modelos utilizados em séries temporais. Primeiramente é proposto o uso de dois algoritmos populares PSO e ACO para otimizar a parametrização de modelos lineares SARIMAX, fazendo busca também por ordem de sazonalidade e variáveis exógenas disponíveis. Logo após são mostradas duas formas de geração de modelos híbridos com MLPs, o primeiro utilizando MLP para fazer correção de resíduo a partir de combinação e modelagem do resíduo e o segundo utilizando conjuntos de MLPs analogamente.

Pelos resultados obtidos, para as localizações de interesse, observa-se que o uso do PSO e ACO para parametrização de modelos SARIMAX é promissor, tendo resultados melhores na definição do modelo do que o algoritmo auto ARIMA, visto que vai além do escopo deste comumente citado. A maior utilidade do proposto para PSO e ACO é a automatização da escolha de variáveis exógenas.

A estratégia adotada para correção do resíduo por otimização de elementos que modelam o erro e também fazem uma combinação não linear deste com a previsão do modelo linear SARIMAX também se mostrou interessante, visto que para todas as séries escolhidas, teve melhor performance de acordo com as métricas adotadas.

No geral, o que foi proposto neste trabalho pode ser facilmente aplicado e utilizado em sistemas autônomos de previsão de series temporais com uso de variáveis exógenas, mais especificamente para previsão de radiação solar.