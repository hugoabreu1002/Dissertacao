\chapter{Introdução}\label{cap:introducao}

\section{Motivação}

Na última década o crescimento da geração fotovoltaica foi expressivo, de acordo com a IEA (Internetional Energy Agency), a produção global saltou de 32TWh em 2010 para mais de 720 TWh em 2020 \cite{ieasolarpvontrack2020}. 

Estimativa de potencial de geração fotovoltaico é um tema que já obteve grande avanço na academia \cite{chin2015cell, jordehi2016parameter, de2017performance}. Os trabalhos levam em consideração a tecnologia utilizada pela célula e modelos de satélite que visam definir os parâmetros físicos de entrada, tais como radiação, temperatura e velocidade do vento \cite{mueller2009cm, huld2012new, amillo2014new, habte2017evaluation}.

A utilização de técnicas de inteligência artificial aplicadas a este tema tem focado na predição temporal de geração \cite{voyant2017machine, wolff2016statistical, li2016hierarchical}. Este tipo de predição é muito útil levando-se em consideração o sistema elétrico completo de uma região ou país, para balanceamento de oferta e demanda, sendo possível uma programabilidade maior para o operador do sistema elétrico, em relação a adequação do uso de outras fontes de energia.

São encontrados alguns métodos de previsão de energia fotovoltaica na literatura, sendo subdivididos de acordo com o horizonte de previsão \cite{mellit2020advanced}. A Tabela \ref{tab:app_forecast} resume os prazos e aplicações para cada método.

\begin{table}[!ht]
\caption{Demonstration of simple table syntax.} 
\label{tab:app_forecast}
\begin{tabular}{c|c|c}
\textbf{Horizonte} & \textbf{Período máximo} & \textbf{Aplicação}                                                                                                               \\ \hline
Curtíssimo prazo   & Minutos                 & \begin{tabular}[c]{@{}c@{}}controle e gestão de sistemas fotovoltaicos\\microredes\\mercado de eletricidade\end{tabular}         \\ \hline
Curto prazo        & 72h                     & \begin{tabular}[c]{@{}c@{}}controle das operações do sistema de potência\\despacho econômico\\comprometimento da unidade\end{tabular} \\ \hline
Médio prazo        & Semanas                  & manutenção e planejamento de usinas fotovoltaicas \\ \hline
Longo prazo        & Anos                     & manutenção e planejamento de usinas fotovoltaicas  \\ \hline                                               
\end{tabular}
\source{Autor.}
\end{table}

Este trabalho busca unir de uma melhor forma o uso da ciência de dados e algoritmos de aprendizado de máquina para estimação de potencial de geração em sítios, levando a otimização da escolha utilizando dados reais de geração e dados de simulação, levando em consideração diversas bases de dados diferentes em um estudo de caso para a região nordeste do Brasil.


\section{Objetivos}

\subsection{Objetivo Geral}

Criar modelos de auto ML capazes de se adaptar e fazer a previsão de séries temporais de irradiação e geração fotovoltaica.

\subsection{Objetivos específicos}

\begin{itemize}
    \item 
\end{itemize}

\section{Estado-da-arte}

Modelos híbridos, em que se utiliza da modelagem linear dada por um ARIMA ou variante já foi muito utilizada e discutida na literatura \cite{zhang2003time, khashei2010artificial, babu2014moving, de2014hybrid, de2016hybrid, domingos2019intelligent}.

\section{Estrutura da dissertação}

A organização das seções deste trabalho se dá da seguinte forma. Primeiramente será apresentada a teoria por trás da previsão de séries temporais utilizando modelos lineares \ref{cap:series_temp}. %No capítulo seguinte \ref{cap:ml_models} são apresentados modelos de machine learning que serão utilizados mais a frente para correção do resíduo.
No capítulo \ref{cap:dados_prep} será discutido os dados de series temporais horárias obtidos a partir da base de dados do INMET [CITATION NEEDED]. Em seguida no capítulo \ref{cap:mod_props} são apresentados os modelos híbridos que utilizam dos algorítmos de machine learning apresentados no capítulo antecedente.